% Options for packages loaded elsewhere
\PassOptionsToPackage{unicode}{hyperref}
\PassOptionsToPackage{hyphens}{url}
%
\documentclass[
]{article}
\usepackage{amsmath,amssymb}
\usepackage{lmodern}
\usepackage{iftex}
\ifPDFTeX
  \usepackage[T1]{fontenc}
  \usepackage[utf8]{inputenc}
  \usepackage{textcomp} % provide euro and other symbols
\else % if luatex or xetex
  \usepackage{unicode-math}
  \defaultfontfeatures{Scale=MatchLowercase}
  \defaultfontfeatures[\rmfamily]{Ligatures=TeX,Scale=1}
\fi
% Use upquote if available, for straight quotes in verbatim environments
\IfFileExists{upquote.sty}{\usepackage{upquote}}{}
\IfFileExists{microtype.sty}{% use microtype if available
  \usepackage[]{microtype}
  \UseMicrotypeSet[protrusion]{basicmath} % disable protrusion for tt fonts
}{}
\makeatletter
\@ifundefined{KOMAClassName}{% if non-KOMA class
  \IfFileExists{parskip.sty}{%
    \usepackage{parskip}
  }{% else
    \setlength{\parindent}{0pt}
    \setlength{\parskip}{6pt plus 2pt minus 1pt}}
}{% if KOMA class
  \KOMAoptions{parskip=half}}
\makeatother
\usepackage{xcolor}
\usepackage[margin=1in]{geometry}
\usepackage{color}
\usepackage{fancyvrb}
\newcommand{\VerbBar}{|}
\newcommand{\VERB}{\Verb[commandchars=\\\{\}]}
\DefineVerbatimEnvironment{Highlighting}{Verbatim}{commandchars=\\\{\}}
% Add ',fontsize=\small' for more characters per line
\usepackage{framed}
\definecolor{shadecolor}{RGB}{248,248,248}
\newenvironment{Shaded}{\begin{snugshade}}{\end{snugshade}}
\newcommand{\AlertTok}[1]{\textcolor[rgb]{0.94,0.16,0.16}{#1}}
\newcommand{\AnnotationTok}[1]{\textcolor[rgb]{0.56,0.35,0.01}{\textbf{\textit{#1}}}}
\newcommand{\AttributeTok}[1]{\textcolor[rgb]{0.77,0.63,0.00}{#1}}
\newcommand{\BaseNTok}[1]{\textcolor[rgb]{0.00,0.00,0.81}{#1}}
\newcommand{\BuiltInTok}[1]{#1}
\newcommand{\CharTok}[1]{\textcolor[rgb]{0.31,0.60,0.02}{#1}}
\newcommand{\CommentTok}[1]{\textcolor[rgb]{0.56,0.35,0.01}{\textit{#1}}}
\newcommand{\CommentVarTok}[1]{\textcolor[rgb]{0.56,0.35,0.01}{\textbf{\textit{#1}}}}
\newcommand{\ConstantTok}[1]{\textcolor[rgb]{0.00,0.00,0.00}{#1}}
\newcommand{\ControlFlowTok}[1]{\textcolor[rgb]{0.13,0.29,0.53}{\textbf{#1}}}
\newcommand{\DataTypeTok}[1]{\textcolor[rgb]{0.13,0.29,0.53}{#1}}
\newcommand{\DecValTok}[1]{\textcolor[rgb]{0.00,0.00,0.81}{#1}}
\newcommand{\DocumentationTok}[1]{\textcolor[rgb]{0.56,0.35,0.01}{\textbf{\textit{#1}}}}
\newcommand{\ErrorTok}[1]{\textcolor[rgb]{0.64,0.00,0.00}{\textbf{#1}}}
\newcommand{\ExtensionTok}[1]{#1}
\newcommand{\FloatTok}[1]{\textcolor[rgb]{0.00,0.00,0.81}{#1}}
\newcommand{\FunctionTok}[1]{\textcolor[rgb]{0.00,0.00,0.00}{#1}}
\newcommand{\ImportTok}[1]{#1}
\newcommand{\InformationTok}[1]{\textcolor[rgb]{0.56,0.35,0.01}{\textbf{\textit{#1}}}}
\newcommand{\KeywordTok}[1]{\textcolor[rgb]{0.13,0.29,0.53}{\textbf{#1}}}
\newcommand{\NormalTok}[1]{#1}
\newcommand{\OperatorTok}[1]{\textcolor[rgb]{0.81,0.36,0.00}{\textbf{#1}}}
\newcommand{\OtherTok}[1]{\textcolor[rgb]{0.56,0.35,0.01}{#1}}
\newcommand{\PreprocessorTok}[1]{\textcolor[rgb]{0.56,0.35,0.01}{\textit{#1}}}
\newcommand{\RegionMarkerTok}[1]{#1}
\newcommand{\SpecialCharTok}[1]{\textcolor[rgb]{0.00,0.00,0.00}{#1}}
\newcommand{\SpecialStringTok}[1]{\textcolor[rgb]{0.31,0.60,0.02}{#1}}
\newcommand{\StringTok}[1]{\textcolor[rgb]{0.31,0.60,0.02}{#1}}
\newcommand{\VariableTok}[1]{\textcolor[rgb]{0.00,0.00,0.00}{#1}}
\newcommand{\VerbatimStringTok}[1]{\textcolor[rgb]{0.31,0.60,0.02}{#1}}
\newcommand{\WarningTok}[1]{\textcolor[rgb]{0.56,0.35,0.01}{\textbf{\textit{#1}}}}
\usepackage{longtable,booktabs,array}
\usepackage{calc} % for calculating minipage widths
% Correct order of tables after \paragraph or \subparagraph
\usepackage{etoolbox}
\makeatletter
\patchcmd\longtable{\par}{\if@noskipsec\mbox{}\fi\par}{}{}
\makeatother
% Allow footnotes in longtable head/foot
\IfFileExists{footnotehyper.sty}{\usepackage{footnotehyper}}{\usepackage{footnote}}
\makesavenoteenv{longtable}
\usepackage{graphicx}
\makeatletter
\def\maxwidth{\ifdim\Gin@nat@width>\linewidth\linewidth\else\Gin@nat@width\fi}
\def\maxheight{\ifdim\Gin@nat@height>\textheight\textheight\else\Gin@nat@height\fi}
\makeatother
% Scale images if necessary, so that they will not overflow the page
% margins by default, and it is still possible to overwrite the defaults
% using explicit options in \includegraphics[width, height, ...]{}
\setkeys{Gin}{width=\maxwidth,height=\maxheight,keepaspectratio}
% Set default figure placement to htbp
\makeatletter
\def\fps@figure{htbp}
\makeatother
\setlength{\emergencystretch}{3em} % prevent overfull lines
\providecommand{\tightlist}{%
  \setlength{\itemsep}{0pt}\setlength{\parskip}{0pt}}
\setcounter{secnumdepth}{-\maxdimen} % remove section numbering
\ifLuaTeX
  \usepackage{selnolig}  % disable illegal ligatures
\fi
\IfFileExists{bookmark.sty}{\usepackage{bookmark}}{\usepackage{hyperref}}
\IfFileExists{xurl.sty}{\usepackage{xurl}}{} % add URL line breaks if available
\urlstyle{same} % disable monospaced font for URLs
\hypersetup{
  pdftitle={BDA project 2022-2023},
  pdfauthor={G.R. van der Ploeg, J.A. Westerhuis, A.U.S. Heintz-Buschart, A.K. Smilde},
  hidelinks,
  pdfcreator={LaTeX via pandoc}}

\title{BDA project 2022-2023}
\author{G.R. van der Ploeg, J.A. Westerhuis, A.U.S. Heintz-Buschart,
A.K. Smilde}
\date{09/01/2023 - 20/01/2023}

\begin{document}
\maketitle

{
\setcounter{tocdepth}{3}
\tableofcontents
}
\newpage

\hypertarget{bda-project-2022-2023}{%
\section{BDA project 2022-2023}\label{bda-project-2022-2023}}

\hypertarget{project-description}{%
\subsection{Project description}\label{project-description}}

Welcome to the BDA project 2022-2023 syllabus. In this document you will
find all the information you need to do your BDA project.\\

During the project, we will use a dataset produced by Caldana et
al.~(2011). During the ``do-it-yourself'' part of the first 6 practical
sessions you should apply the new data analysis method that you have
learned on this dataset. Our aim is that you will learn more about the
methods and their comparison by applying them to the same data. Over
time you will learn what the properties of your data are, what questions
can be answered by the different methods, what samples tend to cluster
together, and what metabolites or genes are important to distinguish the
conditions. Learning the properties of your data is very helpful when
you are applying a new method.

\hypertarget{project-goals}{%
\subsubsection{Project goals}\label{project-goals}}

The following goals are defined for this project:

\begin{enumerate}
\def\labelenumi{\arabic{enumi}.}
\tightlist
\item
  You know the origin of the data and the specific properties of the
  data.\\
\item
  You are able to apply the methods, and you are able to interpret the
  results.\\
\item
  You comprehend the pitfalls of multivariate data and validation
  strategies to prevent overfit.\\
\item
  You are able to critically review data analysis applications of the
  above mentioned methods.\\
\end{enumerate}

Keep in mind that the assignments are intended to assess whether
you/your group has reached these goals. We hope this helps in answering
the questions.

\hypertarget{groups}{%
\subsubsection{Groups}\label{groups}}

We encourage you to do the project in pairs. This is because many of the
open-ended questions we ask during your project are open for debate. In
such cases it is really helpful to have a partner who you can discuss
with about what the next step should be. Also you will be able to run
RStudio on two computers at the same time, which will help you in
splitting up the tasks and completing the assignments faster overall.
Please supply your names in the code block below.

\begin{Shaded}
\begin{Highlighting}[]
\NormalTok{name1 }\OtherTok{=} \StringTok{"John Williams"}
\NormalTok{name2 }\OtherTok{=} \StringTok{"Mary Harris Jones"}
\end{Highlighting}
\end{Shaded}

\hypertarget{submission-and-grading}{%
\subsubsection{Submission and grading}\label{submission-and-grading}}

You will hand in your project markdown file and knitted .pdf files in
the first and second week. In the first week you will get an indicative
pass/fail grade and some feedback to help you progress. In the second
week you will receive a full grade and some feedback. The deadline for
the first week submission is Monday January 16th 11:00, for which you
need to complete Assignments 1 up to and including Assignment 3. The
deadline for the second week submission is Monday January 23rd 11:00,
for which you will need to complete Assignment 1 up to and including
Assignment 6. Note that you can use the week 1 feedback to improve your
answers from Assignments 1-3 before the second submission! Whichever
version of Assignments 1-3 will give you the highest grade will be used
for grading in week 2.

The project grade and the exam grade will be combined to produce the
final grade of the course. The project grade (P) will have weight 1 and
the exam grade (E) will have weight 2. Both grades have to be equal to
or higher than 5.0. If you fail the project but not the exam, you will
have to do a re-take of the project to finish the course.

\(finalGrade = \frac{1*P+2*E}{3}\)

Please refer to the course book for more information on the BDA course
organisation.

Grading of the project will be done using a rubric for each assignment.
An overview of the number of points you can get for every assignment is
given below. Every question states how many points you can get for it.

\begin{longtable}[]{@{}cc@{}}
\toprule()
Assignment & Number of points \\
\midrule()
\endhead
Assignment 1: Linear Algebra & 33 \\
Assignment 2: Gene expression analysis & 33 \\
Assignment 3: Principal Component Analysis & 34 \\
Assignment 4: Clustering & 34 \\
Assignment 5: Classification & 33 \\
Assignment 6: ASCA & 33 \\
Total & 200 \\
\bottomrule()
\end{longtable}

It is crucial for our graders to understand what you have done and why.
Please elaborate in the supplied text boxes below each question what
your reasoning is for making a certain step. To make your plots eligible
for grading, you should add a clear x-axis label, y-axis label and
title, as well as a clear description of what you have done to make your
plot. When writing your code in the code blocks, add comments clarifying
what you are doing! You can do this by adding a hash tag (``\#'') before
the comment.

\hypertarget{the-caldana-data}{%
\subsubsection{The Caldana data}\label{the-caldana-data}}

In this project we will use a dataset produced by Caldana et al.~(2011).
The paper can be found on Canvas, and we highly recommend you to read it
before starting the project. You don't need to understand all parts of
the paper, but it's important to focus on the experimental design, the
computational methods used, and the biology so you can understand your
own results better.

The Caldana et al.~(2011) paper describes an experiment where
\emph{Arabidopsis thaliana} plants were grown under normal temperature
(21\(^\circ\)C) and light conditions (150 \(\mu\)E) (see figure 1).
After some time, they were moved to a different temperature and/or light
condition. The conditions are as follows: the base condition (encoded
21-L), 21 \(^\circ\)C and bright light (21-HL), 32 \(^\circ\)C and
normal light (32-L), 4 \(^\circ\)C and low light (4-L), 21 \(^\circ\)C
and low light (21-LL), 4 \(^\circ\)C in the dark (4-D), 21 \(^\circ\)C
in the dark (21-D) and 32 \(^\circ\)C in the dark (32-D). At every
timepoint, the rosetta leaves of six new plants were sampled for
transcriptomics and metabolomics analysis. This way the adaptation of
the plants to the new conditions could be tracked. Samples were taken at
18 timepoints: 20, 40, \ldots, 360 minutes (and the control timepoint
0).

\begin{figure}

{\centering \includegraphics[width=1\linewidth,height=1\textheight]{./new experimental design} 

}

\caption{Overview of the experimental design as described in Caldana et al. (2011). Arabidopsis thaliana plants were grown under normal temperature (21 degrees) and light conditions (150 uE). After some time, they were moved to a different temperature and/or light condition. The conditions are as follows: the base condition (encoded 21-L), 21 degrees and bright light (21-HL), 32 degrees and normal light (32-L), 4 degrees and low light (4-L), 21 degrees and low light (21-LL), 4 degrees in the dark (4-D), 21 degrees in the dark (21-D) and 32 degrees in the dark (32-D). At every timepoint, the rosetta leaves of six new plants were samples for transcriptomics and metabolomics analysis. This way the adaptation of the plants to the new conditions could be tracked. Samples were taken at 18 timepoints.}\label{fig:unnamed-chunk-1}
\end{figure}

Data have been obtained on 92 metabolites and 15047 unique genes. In
total, samples for 8 conditions x 18 timepoints were measured, plus the
baseline condition at timepoint 0. The replicates were summarized per
condition-timepoint combination. Both datasets have been transformed to
a log-scale as well. In the further project description, we will refer
to the condition-timepoint combinations as ``samples''. Also note that
the 21-L samples are from the plants being kept at the baseline
condition throughout.

During the project we will work a lot with R markdown coding blocks such
as the one below. These are intended for you to use to complete the
assignments. During knitting, these code blocks will automatically be
run and printed in your output document. If you have any comments to
give on what you have done, there are text blocks below each code block
where you can elaborate.

The metabolomics and expression data are supplied in two files in
./Data/ in your unzipped BDA project folder. As your first step in
understanding this data, we will load it into R for you and show you
what the tables look like. We leave the rest of the exploration to you.

\begin{Shaded}
\begin{Highlighting}[]
\NormalTok{df\_metabolomics }\OtherTok{=} \FunctionTok{read.csv}\NormalTok{(}\StringTok{"./Data/Caldana\_et\_al\_metabolomics\_dataset.csv"}\NormalTok{)}
\NormalTok{df\_expression }\OtherTok{=} \FunctionTok{read.csv}\NormalTok{(}\StringTok{"./Data/Caldana\_et\_al\_expression\_dataset.csv"}\NormalTok{)}
\end{Highlighting}
\end{Shaded}

\begin{Shaded}
\begin{Highlighting}[]
\FunctionTok{print}\NormalTok{(df\_metabolomics[}\DecValTok{1}\SpecialCharTok{:}\DecValTok{4}\NormalTok{, }\FunctionTok{c}\NormalTok{(}\DecValTok{2}\NormalTok{,}\DecValTok{3}\NormalTok{,}\DecValTok{6}\NormalTok{,}\DecValTok{7}\NormalTok{,}\DecValTok{8}\NormalTok{)])}
\end{Highlighting}
\end{Shaded}

\begin{verbatim}
##   condition time    Alanine   Arabinose    Arabitol
## 1      21-D  100 -0.2299462  0.12785335  0.33647383
## 2      21-D  120 -0.1992554 -0.03012810 -0.05989591
## 3      21-D  140 -0.2964217 -0.04493523 -0.05922931
## 4      21-D  160 -0.2047872  0.07633106 -0.25555183
\end{verbatim}

Above, a table is shown of a part of the metabolomics dataset. You can
see some values for every metabolite: Alanine, Arabinose and Arabitol.
Additionally, you can see the condition of the sample, in this case
21-D, meaning the plant was at 21 degrees in the dark. The timepoint is
also shown. Every value that you see is the log2 of the area under the
peak of the MS spectrum for that metabolite in that sample, which
reflects the amount of that metabolite. If you're interested in how that
works, we recommend searching the internet for ``GC-MS metabolite
quantification''. You do not need to know how GC-MS works for our
questions and exam.

\begin{Shaded}
\begin{Highlighting}[]
\FunctionTok{print}\NormalTok{(df\_expression[}\DecValTok{1}\SpecialCharTok{:}\DecValTok{4}\NormalTok{, }\DecValTok{2}\SpecialCharTok{:}\DecValTok{7}\NormalTok{])}
\end{Highlighting}
\end{Shaded}

\begin{verbatim}
##   condition time AT2G26550 AT2G26570 AT2G26580 AT2G26360
## 1      21-L    0      7.90      8.14      6.70      2.81
## 2      21-L   20      8.14      8.12      6.00      2.48
## 3      21-L   40      8.08      8.18      6.71      2.51
## 4      21-L   60      8.08      8.19      6.43      2.47
\end{verbatim}

Here a table is shown of a part of the gene expression dataset. You can
see some values for some Arabidopsis thaliana genes: AT2G26550,
AT2G26570, AT2G26580, and AT2G26360. When you look up these genes on
\href{https://www.arabidopsis.org/index.jsp}{TAIR}, you can see that
they encode a haeme oxygenase-like protein, a coiled-coil protein that
moves chloroplasts, a transcription factor and an unnamed protein,
respectively. As before, you can see the condition and time combination
for the sample as well. Every value that you see is a normalized
microarray intensity value, which reflects the transcript abundance of
that gene in the sample.

\hypertarget{plant-conditions-to-compare}{%
\subsubsection{Plant conditions to
compare}\label{plant-conditions-to-compare}}

During the assignments, you will often be asked to compare two plant
conditions (see Figure 1). Please discuss which two conditions you want
to compare consistently throughout your project, and state them below in
the code block. An example choice is given. Use the supplied text box to
explain why you want to investigate these two conditions.

\begin{Shaded}
\begin{Highlighting}[]
\NormalTok{condition1 }\OtherTok{=} \StringTok{"21{-}HL"}
\NormalTok{condition2 }\OtherTok{=} \StringTok{"21{-}D"}
\end{Highlighting}
\end{Shaded}

\hypertarget{explain-your-choice-here.}{%
\paragraph{\texorpdfstring{Explain your choice here.\\
}{Explain your choice here. }}\label{explain-your-choice-here.}}

\hypertarget{end-of-text-block.}{%
\paragraph{End of text block.}\label{end-of-text-block.}}

\hypertarget{help}{%
\subsubsection{Help}\label{help}}

As any programmer will tell you, looking stuff up on Google is a major
part of any project. Hence looking up specific R programming questions
may be helpful in progressing in your project assignments. Coding
websites such as \href{https://stackoverflow.com/}{StackOverflow} are
particularly helpful.

Specific questions on how functions work in R can be answered using the
R documentation. Say you want to understand how you can run the singular
value decomposition function svd(). On the right side of your screen you
can find a ``help'' tab where you can search for the function that you
want to use. You can type ``svd'' in the search field there to reach the
documentation of the svd() function. Additionally, you can type ``?svd''
in your console window at the bottom of your screen to automatically
search for the documentation of your supplied function.

This project is a new addition to the BDA course, and will replace the R
test from previous years. If you have done BDA in previous years, the
rules and regulations regarding grades and exemptions are the same as
they used to be for the R test. If you have passed the R-test last year,
you do not have to hand in the project assignments again this year. If
you have failed the R-test last year, you will need to do the project
this year instead.

If things are unclear to you or you have any other question, feel free
to ask any of your lecturers or teaching assistants (TAs):\\
- Johan Westerhuis (Course coordinator,
\href{mailto:j.a.westerhuis@uva.nl}{\nolinkurl{j.a.westerhuis@uva.nl}})\\
- Anna Heintz Buschart (Lecturer,
\href{mailto:a.u.s.heintzbuschart@uva.nl}{\nolinkurl{a.u.s.heintzbuschart@uva.nl}})\\
- Roel van der Ploeg (Project coordinator,
\href{mailto:g.r.ploeg@uva.nl}{\nolinkurl{g.r.ploeg@uva.nl}})\\
- Archontis Goumagias (TA,
\href{mailto:a.goumagias@student.vu.nl}{\nolinkurl{a.goumagias@student.vu.nl}})\\
- Lucas Jansen (TA,
\href{mailto:l.s.jansen@student.vu.nl}{\nolinkurl{l.s.jansen@student.vu.nl}})\\
- Alex van Kaam (TA,
\href{mailto:a.t.van.kaam@student.vu.nl}{\nolinkurl{a.t.van.kaam@student.vu.nl}})\\

\hypertarget{assignments}{%
\subsection{Assignments}\label{assignments}}

Below are the packages we will use throughout the course. Do not edit
this code block!

\begin{Shaded}
\begin{Highlighting}[]
\ControlFlowTok{if}\NormalTok{(}\SpecialCharTok{!}\StringTok{"umap"} \SpecialCharTok{\%in\%} \FunctionTok{installed.packages}\NormalTok{()[,}\DecValTok{1}\NormalTok{])\{}
  \FunctionTok{install.packages}\NormalTok{(}\StringTok{"umap"}\NormalTok{)}
\NormalTok{\}}

\ControlFlowTok{if}\NormalTok{(}\SpecialCharTok{!}\StringTok{"MASS"} \SpecialCharTok{\%in\%} \FunctionTok{installed.packages}\NormalTok{()[,}\DecValTok{1}\NormalTok{])\{}
  \FunctionTok{install.packages}\NormalTok{(}\StringTok{"MASS"}\NormalTok{)}
\NormalTok{\}}

\ControlFlowTok{if}\NormalTok{(}\SpecialCharTok{!}\StringTok{"gtools"} \SpecialCharTok{\%in\%} \FunctionTok{installed.packages}\NormalTok{()[,}\DecValTok{1}\NormalTok{])\{}
  \FunctionTok{install.packages}\NormalTok{(}\StringTok{"gtools"}\NormalTok{)}
\NormalTok{\}}

\ControlFlowTok{if}\NormalTok{(}\SpecialCharTok{!}\StringTok{"stringr"} \SpecialCharTok{\%in\%} \FunctionTok{installed.packages}\NormalTok{()[,}\DecValTok{1}\NormalTok{])\{}
  \FunctionTok{install.packages}\NormalTok{(}\StringTok{"stringr"}\NormalTok{)}
\NormalTok{\}}

\FunctionTok{library}\NormalTok{(umap)}
\FunctionTok{library}\NormalTok{(MASS)}
\FunctionTok{library}\NormalTok{(gtools)}
\FunctionTok{library}\NormalTok{(stringr)}
\FunctionTok{source}\NormalTok{(}\StringTok{"./heatmap.2a.R"}\NormalTok{)}
\end{Highlighting}
\end{Shaded}

\hypertarget{assignment-1-linear-algebra-johan-westerhuis-09-01-2023}{%
\subsubsection{Assignment 1: Linear Algebra (Johan Westerhuis,
09-01-2023)}\label{assignment-1-linear-algebra-johan-westerhuis-09-01-2023}}

In this assignment, you will start your exploration of the data by
checking various summaries per row and per column of your data tables.
You will also make a few histograms and inspect them to check the
comparability of our samples and variables. Finally you will center and
scale your data for use in the other assignments.

\begin{enumerate}
\def\labelenumi{\Alph{enumi})}
\tightlist
\item
  Import both datasets. You will see that the first column is full of
  sample names. Change the import code so it automatically uses that
  column as the row names. (Hint: use read.csv(), use the row.names
  argument to specify the column. Type ``?read.csv'' in your console
  window for more info.) \textbf{{[}1 point{]}}
\end{enumerate}

\hypertarget{put-any-text-you-may-want-to-type-here.}{%
\paragraph{Put any text you may want to type
here.}\label{put-any-text-you-may-want-to-type-here.}}

\hypertarget{end-of-text-block.-1}{%
\paragraph{End of text block.}\label{end-of-text-block.-1}}

\begin{enumerate}
\def\labelenumi{\Alph{enumi})}
\setcounter{enumi}{1}
\tightlist
\item
  Create a version of both datasets that only contains numerical data.
  That means you need to remove the ``condition'' and ``time'' columns.
  \textbf{{[}1 point{]}}
\end{enumerate}

\hypertarget{put-any-text-you-may-want-to-type-here.-1}{%
\paragraph{Put any text you may want to type
here.}\label{put-any-text-you-may-want-to-type-here.-1}}

\hypertarget{end-of-text-block.-2}{%
\paragraph{End of text block.}\label{end-of-text-block.-2}}

\begin{enumerate}
\def\labelenumi{\Alph{enumi})}
\setcounter{enumi}{2}
\tightlist
\item
  Calculate the total amounts per row and per column of both datasets.
  Use your numerical-only datasets for this. Plot the row and column
  sums in a histogram per dataset. What do the plots indicate about the
  comparability of the samples and variables? (Hint: you can use
  rowSums() and colSums() for this.) \textbf{{[}10 points{]}}
\end{enumerate}

\hypertarget{write-down-your-observations-here.}{%
\paragraph{Write down your observations
here.}\label{write-down-your-observations-here.}}

\hypertarget{end-of-text-block.-3}{%
\paragraph{End of text block.}\label{end-of-text-block.-3}}

\begin{enumerate}
\def\labelenumi{\Alph{enumi})}
\setcounter{enumi}{3}
\tightlist
\item
  Calculate the column mean and column standard deviation for each
  dataset. Make a histogram of your means and a histogram of your
  standard deviations for each dataset. Explain what the histograms
  describe in terms of comparability of the variables. (Hint: you can
  use apply() to calculate the mean and standard deviations per column.)
  \textbf{{[}10 points{]}}
\end{enumerate}

\hypertarget{put-any-text-you-may-want-to-type-here.-2}{%
\paragraph{Put any text you may want to type
here.}\label{put-any-text-you-may-want-to-type-here.-2}}

\hypertarget{end-of-text-block.-4}{%
\paragraph{End of text block.}\label{end-of-text-block.-4}}

\begin{enumerate}
\def\labelenumi{\Alph{enumi})}
\setcounter{enumi}{4}
\tightlist
\item
  Extract the amount of Glycine for the metabolomics 21-L samples for
  all timepoints. Save it in a variable. Plot the amount of Glycine over
  time in the input data, after you centered your variable, and after
  you centered \& scaled your variable. Check what has happened to the
  mean and the standard deviation of your variable in each case. Explain
  what centering and/or scaling does to your data. \textbf{{[}7
  points{]}}
\end{enumerate}

\hypertarget{put-any-text-you-may-want-to-type-here.-3}{%
\paragraph{Put any text you may want to type
here.}\label{put-any-text-you-may-want-to-type-here.-3}}

\hypertarget{end-of-text-block.-5}{%
\paragraph{\texorpdfstring{End of text block.\\
}{End of text block. }}\label{end-of-text-block.-5}}

\begin{enumerate}
\def\labelenumi{\Alph{enumi})}
\setcounter{enumi}{5}
\item
  Create a centered version of both datasets. You can do this by
  removing the column mean from every column. (Hint: consider using the
  sweep() function, you can use the means you have calculated in
  Assignment 1D for this.) \textbf{{[}2 points{]}}
\item
  Create a centered \& scaled version of both datasets. You can do this
  by dividing a centered column by its standard deviation. This is also
  known as autoscaling. (Hint: consider using the sweep() function, you
  can use the standard deviations you have calculated in Assignment 1D
  for this.) \textbf{{[}2 points{]}}
\end{enumerate}

\textbf{Note}: You have created new versions of the gene expression and
metabolomics datasets. In the rest of the assignments we expect you to
select the appropriate version of a dataset to use for a given method.

\hypertarget{put-any-text-you-may-want-to-type-here.-4}{%
\paragraph{Put any text you may want to type
here.}\label{put-any-text-you-may-want-to-type-here.-4}}

\hypertarget{end-of-text-block.-6}{%
\paragraph{\texorpdfstring{End of text block.\\
}{End of text block. }}\label{end-of-text-block.-6}}

\hfill\break
\textbf{This concludes Assignment 1. Make sure you save your .Rmd file
and workspace so you can continue in your next project session.}

\hypertarget{assignment-2-gene-expression-analysis-douwe-molenaar-12-01-2023}{%
\subsubsection{Assignment 2: Gene expression analysis (Douwe Molenaar,
12-01-2023)}\label{assignment-2-gene-expression-analysis-douwe-molenaar-12-01-2023}}

In this assignment we will consider the gene expression data without
centering or scaling it. We will perform a differential expression
analysis to compare your two chosen plant conditions. While we ask you
to do this using t-tests, please be aware that this is not really
appropriate given the experimental design of the study. This is because
the data points are not randomly drawn from a population, as they come
from a time series.

\begin{enumerate}
\def\labelenumi{\Alph{enumi})}
\tightlist
\item
  Perform a t-test of the expression of AT2G20560 between your two
  chosen conditions. What does this p-value mean? Is this gene
  differentially expressed between your two conditions? \textbf{{[}4
  points{]}}
\end{enumerate}

\hypertarget{put-any-text-you-may-want-to-type-here.-5}{%
\paragraph{Put any text you may want to type
here.}\label{put-any-text-you-may-want-to-type-here.-5}}

\hypertarget{end-of-text-block.-7}{%
\paragraph{End of text block.}\label{end-of-text-block.-7}}

\begin{enumerate}
\def\labelenumi{\Alph{enumi})}
\setcounter{enumi}{1}
\tightlist
\item
  Perform a t-test for every gene in the gene expression dataset for
  your chosen conditions. Collect the p-values in a vector. (Hint:
  consider using apply() or sapply().) \textbf{{[}3 points{]}}
\end{enumerate}

\hypertarget{put-any-text-you-may-want-to-type-here.-6}{%
\paragraph{Put any text you may want to type
here.}\label{put-any-text-you-may-want-to-type-here.-6}}

\hypertarget{end-of-text-block.-8}{%
\paragraph{End of text block.}\label{end-of-text-block.-8}}

\begin{enumerate}
\def\labelenumi{\Alph{enumi})}
\setcounter{enumi}{2}
\tightlist
\item
  Make a histogram of all the p-values. Explain what the outcome means
  for the significance of your t-tests between your two chosen
  conditions. Is it possible for a gene to not be differentially
  expressed, and to still have a low p-value? \textbf{{[}5 points{]}}
\end{enumerate}

\hypertarget{put-any-text-you-may-want-to-type-here.-7}{%
\paragraph{Put any text you may want to type
here.}\label{put-any-text-you-may-want-to-type-here.-7}}

\hypertarget{end-of-text-block.-9}{%
\paragraph{End of text block.}\label{end-of-text-block.-9}}

\begin{enumerate}
\def\labelenumi{\Alph{enumi})}
\setcounter{enumi}{3}
\tightlist
\item
  In your exercises on RNAseq data analysis, you have computed the
  log(ratio) of gene expression using the total RNA counts of a gene
  between two conditions. Our gene expression data is already normalized
  and log transformed, so we can instead calculate the fold change using
  the averages between the two conditions.\\
  Calculate the fold change (FC) of AT2G20560 between your two chosen
  conditions. This means that you need to determine the mean of
  AT2G20560 in each condition separately. You can then subtract the mean
  of AT2G20560 in condition 2 from the mean of AT2G20560 in condition 1.
  \textbf{{[}3 points{]}}
\end{enumerate}

\hypertarget{put-any-text-you-may-want-to-type-here.-8}{%
\paragraph{Put any text you may want to type
here.}\label{put-any-text-you-may-want-to-type-here.-8}}

\hypertarget{end-of-text-block.-10}{%
\paragraph{End of text block.}\label{end-of-text-block.-10}}

\begin{enumerate}
\def\labelenumi{\Alph{enumi})}
\setcounter{enumi}{4}
\tightlist
\item
  Calculate the fold change of all genes between your two chosen
  conditions this way. \textbf{{[}3 points{]}}
\end{enumerate}

\hypertarget{put-any-text-you-may-want-to-type-here.-9}{%
\paragraph{Put any text you may want to type
here.}\label{put-any-text-you-may-want-to-type-here.-9}}

\hypertarget{end-of-text-block.-11}{%
\paragraph{End of text block.}\label{end-of-text-block.-11}}

\begin{enumerate}
\def\labelenumi{\Alph{enumi})}
\setcounter{enumi}{5}
\tightlist
\item
  Make a volcano plot using your FC and p-values for every gene.
  Interpret the result. Where are the important differentially expressed
  genes in this plot? Why? (Hint: to make a proper volcano plot you
  still need to do something to your p-values.) \textbf{{[}6 points{]}}
\end{enumerate}

\hypertarget{put-any-text-you-may-want-to-type-here.-10}{%
\paragraph{\texorpdfstring{Put any text you may want to type here.\\
}{Put any text you may want to type here. }}\label{put-any-text-you-may-want-to-type-here.-10}}

\hypertarget{end-of-text-block.-12}{%
\paragraph{End of text block.}\label{end-of-text-block.-12}}

\begin{enumerate}
\def\labelenumi{\Alph{enumi})}
\setcounter{enumi}{6}
\tightlist
\item
  As you may be aware, you have now done over 15.000 t-tests. As such,
  you will need to control for Type I error. In the exercises you did
  before the project part of today's computer practical you have
  calculated the FDR value for any p-value and plotted them. An easy
  function to obtain the FDR values is p.adjust(). Control the type I
  error of your p-values using a false discovery rate. \textbf{{[}2
  points{]}}
\end{enumerate}

\hypertarget{put-any-text-you-may-want-to-type-here.-11}{%
\paragraph{Put any text you may want to type
here.}\label{put-any-text-you-may-want-to-type-here.-11}}

\hypertarget{end-of-text-block.-13}{%
\paragraph{End of text block.}\label{end-of-text-block.-13}}

\begin{enumerate}
\def\labelenumi{\Alph{enumi})}
\setcounter{enumi}{7}
\tightlist
\item
  Find the genes that are differentially expressed by setting an
  ``adjusted'' p-value and FC threshold. Elaborate on your choice of
  threshold. What gene has the lowest ``adjusted'' p-value? What is its
  biological function? Does this result make sense given your choice of
  compared conditions? \textbf{{[}7 points{]}}
\end{enumerate}

\hypertarget{put-any-text-you-may-want-to-type-here.-12}{%
\paragraph{Put any text you may want to type
here.}\label{put-any-text-you-may-want-to-type-here.-12}}

\hypertarget{end-of-text-block.-14}{%
\paragraph{\texorpdfstring{End of text block.\\
}{End of text block. }}\label{end-of-text-block.-14}}

\hfill\break
\textbf{This concludes Assignment 2. Make sure you save your .Rmd file
and workspace so you can continue in your next project session.}

\hypertarget{assignment-3-principal-component-analysis-age-smilde-13-01-2023}{%
\subsubsection{Assignment 3: Principal Component Analysis (Age Smilde,
13-01-2023)}\label{assignment-3-principal-component-analysis-age-smilde-13-01-2023}}

During this Assignment, you will perform Principal Component Analysis on
both the gene expression and metabolomics data containing all
conditions. We will investigate groupings in the score plots, and have a
look at important variables distinguishing your chosen conditions using
the loadings.

\begin{enumerate}
\def\labelenumi{\Alph{enumi})}
\tightlist
\item
  Perform a principal component analysis on the gene expression data:
  Consider which version of the data from Assignment 1 you want to use.
  Make a score plot using the first and second principal component. Show
  the variance explained of each component in your plot labels. Add a
  legend to identify the colour for each condition. Does your plot show
  clear groups? What plant conditions does your PCA plot together in a
  group? Compare your plot with Figure 2 from the paper. (Hint: use the
  legend() function to clarify which colour belongs to which plant
  condition.) \textbf{{[}6 points{]}}
\end{enumerate}

\hypertarget{put-any-text-you-may-want-to-type-here.-13}{%
\paragraph{Put any text you may want to type
here.}\label{put-any-text-you-may-want-to-type-here.-13}}

\hypertarget{end-of-text-block.-15}{%
\paragraph{End of text block.}\label{end-of-text-block.-15}}

\begin{enumerate}
\def\labelenumi{\Alph{enumi})}
\setcounter{enumi}{1}
\tightlist
\item
  Perform a principal component analysis of the metabolomics data:
  Consider which version of the data you want to use. Make a score plot
  using the first and second principal component. Show the variance
  explained by each component in your plot labels. Add a legend to
  identify the colour for each condition. Does your plot show clear
  groups? What plant conditions does your PCA plot together in a group?
  (Hint: use SVD, consider centering and/or scaling your data for PCA
  specifically!) \textbf{{[}6 points{]}}\\
  \strut \\
  \textbf{Note}: your result is likely different from the PCA shown in
  Figure 2 of the paper. This is because the authors have chosen to log2
  transform and subsequently scale their metabolomics data to median 1
  for every column. There is no clear ``right'' or ``wrong'' PCA here.
\end{enumerate}

\hypertarget{put-any-text-you-may-want-to-type-here.-14}{%
\paragraph{Put any text you may want to type
here.}\label{put-any-text-you-may-want-to-type-here.-14}}

\hypertarget{end-of-text-block.-16}{%
\paragraph{End of text block.}\label{end-of-text-block.-16}}

\begin{enumerate}
\def\labelenumi{\Alph{enumi})}
\setcounter{enumi}{2}
\tightlist
\item
  Consider your metabolomics score plot. Which of the first two
  principal components is best at separating your two chosen conditions?
  Plot the loadings of that component in a bar plot. Which metabolite is
  considered the most important? What is the biological role of this
  metabolite? Does this result make biological sense given your choice
  of compared conditions? \textbf{{[}4 points{]}}
\end{enumerate}

\hypertarget{put-any-text-you-may-want-to-type-here.-15}{%
\paragraph{Put any text you may want to type
here.}\label{put-any-text-you-may-want-to-type-here.-15}}

\hypertarget{end-of-text-block.-17}{%
\paragraph{End of text block.}\label{end-of-text-block.-17}}

\begin{enumerate}
\def\labelenumi{\Alph{enumi})}
\setcounter{enumi}{3}
\tightlist
\item
  Now consider your gene expression PCA model. Reconstruct your gene
  expression dataset using the first and second PCs. (Hint: consider the
  way a dataset is decomposed to reconstruct the data.) \textbf{{[}2
  points{]}}
\end{enumerate}

\hypertarget{put-any-text-you-may-want-to-type-here.-16}{%
\paragraph{Put any text you may want to type
here.}\label{put-any-text-you-may-want-to-type-here.-16}}

\hypertarget{end-of-text-block.-18}{%
\paragraph{End of text block.}\label{end-of-text-block.-18}}

\begin{enumerate}
\def\labelenumi{\Alph{enumi})}
\setcounter{enumi}{4}
\tightlist
\item
  Calculate the residuals of your gene expression dataset by subtracting
  the reconstructed data from the input data. Inspect the residuals.
  Report the condition and timepoint that was modeled most poorly. How
  much worse is it modeled than the average? \textbf{{[}4 points{]}}
\end{enumerate}

\hypertarget{put-any-text-you-may-want-to-type-here.-17}{%
\paragraph{Put any text you may want to type
here.}\label{put-any-text-you-may-want-to-type-here.-17}}

\hypertarget{end-of-text-block.-19}{%
\paragraph{End of text block.}\label{end-of-text-block.-19}}

\begin{enumerate}
\def\labelenumi{\Alph{enumi})}
\setcounter{enumi}{5}
\tightlist
\item
  Use your reconstructed gene expression data to calculate the variance
  explained per gene of the entire gene expression PCA model. You can do
  this by dividing the sum of squares of your reconstructed gene by the
  sum of squares of the gene in your input data. Which gene was modeled
  the best? What is its biological role? Does this result make
  biological sense given your choice of compared conditions? (Hint:
  consider using a colSums() approach.) \textbf{{[}2 points{]}}
\end{enumerate}

\hypertarget{put-any-text-you-may-want-to-type-here.-18}{%
\paragraph{Put any text you may want to type
here.}\label{put-any-text-you-may-want-to-type-here.-18}}

\hypertarget{end-of-text-block.-20}{%
\paragraph{End of text block.}\label{end-of-text-block.-20}}

\begin{enumerate}
\def\labelenumi{\Alph{enumi})}
\setcounter{enumi}{6}
\tightlist
\item
  Consider which principal component of the gene expression PCA best
  separates your chosen conditions. Check its loading. Which genes are
  considered important? Compare your result with the set of most
  differentially expressed genes from Assignment 2. What similarities
  and differences do you find? \textbf{{[}4 points{]}}
\end{enumerate}

\hypertarget{put-any-text-you-may-want-to-type-here.-19}{%
\paragraph{Put any text you may want to type
here.}\label{put-any-text-you-may-want-to-type-here.-19}}

\hypertarget{end-of-text-block.-21}{%
\paragraph{End of text block.}\label{end-of-text-block.-21}}

\begin{enumerate}
\def\labelenumi{\Alph{enumi})}
\setcounter{enumi}{7}
\tightlist
\item
  Why would the selected genes from differential expression analysis and
  PCA be different? Explain what the goal is of both methods and thus
  what it means that a variable is selected in either method.
  \textbf{{[}6 points{]}}
\end{enumerate}

\hypertarget{put-any-text-you-may-want-to-type-here.-20}{%
\paragraph{Put any text you may want to type
here.}\label{put-any-text-you-may-want-to-type-here.-20}}

\hypertarget{end-of-text-block.-22}{%
\paragraph{\texorpdfstring{End of text block.\\
}{End of text block. }}\label{end-of-text-block.-22}}

\hfill\break
\textbf{That concludes week 1 of the project! Please hand in your .rmd
and .pdf files on Canvas for your pass/fail grade and feedback. The week
1 deadline is Monday, January 16th 11:00.}

\hypertarget{assignment-4-clustering-johan-westerhuis-16-01-2023}{%
\subsubsection{Assignment 4: Clustering (Johan Westerhuis,
16-01-2023)}\label{assignment-4-clustering-johan-westerhuis-16-01-2023}}

In this assignment, you will focus on clustering your gene expression
and metabolomics data. This will be done using hierarchical clustering
and UMAP methods. You will also inspect the validity of the clustering
results and compare it with the Caldana paper.\\

In the code block below, some functions are defined for you to use in
subsequent questions. You do not need to understand what happens in this
code. Do not edit the code block!

\begin{Shaded}
\begin{Highlighting}[]
\NormalTok{conditions }\OtherTok{=} \FunctionTok{unique}\NormalTok{(df\_expression}\SpecialCharTok{$}\NormalTok{condition)}
\NormalTok{colours }\OtherTok{=} \FunctionTok{c}\NormalTok{(}\StringTok{"grey"}\NormalTok{, }\StringTok{"black"}\NormalTok{, }\StringTok{"cyan"}\NormalTok{, }\StringTok{"blue"}\NormalTok{, }\StringTok{"orange"}\NormalTok{, }\StringTok{"red"}\NormalTok{, }\StringTok{"green"}\NormalTok{, }\StringTok{"yellow"}\NormalTok{)}

\NormalTok{plot\_hclust }\OtherTok{=} \ControlFlowTok{function}\NormalTok{(hclust\_result)\{}
\NormalTok{  dendrogram }\OtherTok{=} \FunctionTok{dendrapply}\NormalTok{(}\FunctionTok{as.dendrogram}\NormalTok{(hclust\_result), labelCol)}
  \FunctionTok{plot}\NormalTok{(dendrogram)}
\NormalTok{\}}

\NormalTok{labelCol }\OtherTok{\textless{}{-}} \ControlFlowTok{function}\NormalTok{(x) \{}
  \ControlFlowTok{if}\NormalTok{ (}\FunctionTok{is.leaf}\NormalTok{(x)) \{}
    
    \DocumentationTok{\#\# fetch label}
\NormalTok{    label }\OtherTok{=} \FunctionTok{attr}\NormalTok{(x, }\StringTok{"label"}\NormalTok{) }
    
    \DocumentationTok{\#\# extract condition}
\NormalTok{    condition }\OtherTok{=} \FunctionTok{str\_split\_fixed}\NormalTok{(label, }\StringTok{"\_"}\NormalTok{, }\DecValTok{2}\NormalTok{)[,}\DecValTok{1}\NormalTok{]}
    
    \DocumentationTok{\#\# set label color}
\NormalTok{    index }\OtherTok{=} \FunctionTok{which}\NormalTok{(conditions }\SpecialCharTok{==}\NormalTok{ condition)}
    \FunctionTok{attr}\NormalTok{(x, }\StringTok{"nodePar"}\NormalTok{) }\OtherTok{=} \FunctionTok{list}\NormalTok{(}\AttributeTok{lab.col =}\NormalTok{ colours[index])}
\NormalTok{  \}}
  \FunctionTok{return}\NormalTok{(x)}
\NormalTok{\}}

\NormalTok{make\_heatmap }\OtherTok{=} \ControlFlowTok{function}\NormalTok{(data, condition1, condition2)\{}
\NormalTok{  referenceSample }\OtherTok{=}\NormalTok{ data[(data}\SpecialCharTok{$}\NormalTok{condition }\SpecialCharTok{==} \StringTok{"21{-}L"}\NormalTok{) }\SpecialCharTok{\&}\NormalTok{ (data}\SpecialCharTok{$}\NormalTok{time }\SpecialCharTok{==} \DecValTok{0}\NormalTok{), }\DecValTok{3}\SpecialCharTok{:}\DecValTok{94}\NormalTok{]}
\NormalTok{  data }\OtherTok{=}\NormalTok{ data[data}\SpecialCharTok{$}\NormalTok{condition }\SpecialCharTok{\%in\%} \FunctionTok{c}\NormalTok{(condition1, condition2),]}
\NormalTok{  conditionLabels }\OtherTok{=}\NormalTok{ data}\SpecialCharTok{$}\NormalTok{condition}
\NormalTok{  timepoints }\OtherTok{=}\NormalTok{ data}\SpecialCharTok{$}\NormalTok{time}
\NormalTok{  data }\OtherTok{=} \FunctionTok{sweep}\NormalTok{(data[,}\DecValTok{3}\SpecialCharTok{:}\DecValTok{94}\NormalTok{], }\DecValTok{2}\NormalTok{, }\FunctionTok{as.numeric}\NormalTok{(referenceSample), }\AttributeTok{FUN=}\StringTok{"{-}"}\NormalTok{)}
  
\NormalTok{  selectMetabolites }\OtherTok{=} \FunctionTok{as.data.frame}\NormalTok{(}\FunctionTok{cbind}\NormalTok{(}\FunctionTok{colnames}\NormalTok{(data), }\FunctionTok{apply}\NormalTok{(}\FunctionTok{abs}\NormalTok{(data), }\DecValTok{2}\NormalTok{, mean)))}
\NormalTok{  selectMetabolites }\OtherTok{=}\NormalTok{ selectMetabolites[}\FunctionTok{order}\NormalTok{(selectMetabolites[,}\DecValTok{2}\NormalTok{], }\AttributeTok{decreasing=}\NormalTok{T),]}
\NormalTok{  selectMetabolites }\OtherTok{=} \FunctionTok{row.names}\NormalTok{(selectMetabolites)[}\DecValTok{1}\SpecialCharTok{:}\DecValTok{20}\NormalTok{]}
  
\NormalTok{  reordering }\OtherTok{=} \FunctionTok{order}\NormalTok{(}\FunctionTok{factor}\NormalTok{(conditionLabels, }\AttributeTok{levels=}\FunctionTok{c}\NormalTok{(}\FunctionTok{c}\NormalTok{(condition1, condition2))), timepoints)}
\NormalTok{  data }\OtherTok{=}\NormalTok{ data[reordering, }\FunctionTok{colnames}\NormalTok{(data) }\SpecialCharTok{\%in\%}\NormalTok{ selectMetabolites]}
\NormalTok{  conditionLabels }\OtherTok{=}\NormalTok{ conditionLabels[reordering]}
  
\NormalTok{  data }\OtherTok{=} \FunctionTok{t}\NormalTok{(data)}
\NormalTok{  colColours }\OtherTok{=} \DecValTok{1}\SpecialCharTok{:}\FunctionTok{ncol}\NormalTok{(data)}

  \ControlFlowTok{for}\NormalTok{(i }\ControlFlowTok{in} \DecValTok{1}\SpecialCharTok{:}\FunctionTok{ncol}\NormalTok{(data))\{}
\NormalTok{  sampleCondition }\OtherTok{=}\NormalTok{ conditionLabels[i]}
\NormalTok{  index }\OtherTok{=} \FunctionTok{which}\NormalTok{(conditions }\SpecialCharTok{==}\NormalTok{ sampleCondition)}
\NormalTok{  colColours[i] }\OtherTok{=}\NormalTok{ colours[index]}
\NormalTok{  \}}

  \FunctionTok{heatmap.2a}\NormalTok{(}\FunctionTok{as.matrix}\NormalTok{(data[,}\FunctionTok{order}\NormalTok{(conditionLabels)]), }\AttributeTok{scale=}\StringTok{"row"}\NormalTok{, }\AttributeTok{hclustfun=}\ControlFlowTok{function}\NormalTok{(x)\{}\FunctionTok{hclust}\NormalTok{(x,}\AttributeTok{method=}\StringTok{"average"}\NormalTok{)\}, }\AttributeTok{Colv=}\ConstantTok{FALSE}\NormalTok{, }\AttributeTok{breaks=}\DecValTok{48}\NormalTok{, }\AttributeTok{col=}\FunctionTok{colorRampPalette}\NormalTok{(}\FunctionTok{c}\NormalTok{(}\StringTok{"red"}\NormalTok{, }\StringTok{"white"}\NormalTok{, }\StringTok{"blue"}\NormalTok{), }\AttributeTok{bias=}\FloatTok{0.75}\NormalTok{, }\AttributeTok{space=}\StringTok{"rgb"}\NormalTok{), }\AttributeTok{trace=}\StringTok{"none"}\NormalTok{, }\AttributeTok{dendrogram=}\StringTok{"row"}\NormalTok{, }\AttributeTok{ColSideColors=}\NormalTok{colColours)}
\NormalTok{\}}
\end{Highlighting}
\end{Shaded}

\begin{enumerate}
\def\labelenumi{\Alph{enumi})}
\tightlist
\item
  Calculate the euclidean distances between the samples in the gene
  expression data (consider which version of the data you wish to use).
  Then, use the distance matrix to create a hierarchical clustering
  using average linkage. Plot the hierarchical clustering using the
  plot\_hclust() function defined in the code block above. An example of
  how you can run this custom function is given in the code block below.
  Compare your result with Figure 3 from the paper. Do you find the same
  clusters as the paper? Describe what plant conditions the dendrogram
  clusters together. (Hint: check the dist() and hclust() function
  documentation.) \textbf{{[}7 points{]}}
\end{enumerate}

\begin{Shaded}
\begin{Highlighting}[]
\CommentTok{\#plot\_hclust(hclust\_result)}
\end{Highlighting}
\end{Shaded}

\hypertarget{put-any-text-you-may-want-to-type-here.-21}{%
\paragraph{Put any text you may want to type
here.}\label{put-any-text-you-may-want-to-type-here.-21}}

\hypertarget{end-of-text-block.-23}{%
\paragraph{End of text block.}\label{end-of-text-block.-23}}

\begin{enumerate}
\def\labelenumi{\Alph{enumi})}
\setcounter{enumi}{1}
\tightlist
\item
  Calculate the euclidean distances between the samples in the
  metabolomics data (consider which version of the data you wish to
  use). Use the distance matrix to create a hierarchical clustering
  using average linkage. Plot the hierarchical clustering using the
  plot\_hclust() function. Compare your result with Figure 3 from the
  paper. Do you find the same clusters as the paper? Describe what plant
  conditions the dendrogram clusters together. \textbf{{[}7 points{]}}
\end{enumerate}

\hypertarget{put-any-text-you-may-want-to-type-here.-22}{%
\paragraph{Put any text you may want to type
here.}\label{put-any-text-you-may-want-to-type-here.-22}}

\hypertarget{end-of-text-block.-24}{%
\paragraph{End of text block.}\label{end-of-text-block.-24}}

\begin{enumerate}
\def\labelenumi{\Alph{enumi})}
\setcounter{enumi}{2}
\tightlist
\item
  Compare the hierarchical clusterings of your metabolomics and gene
  expression data. Do they cluster in the same way? Explain what
  similarities and/or differences between the datasets could cause this
  result. \textbf{{[}6 points{]}}
\end{enumerate}

\hypertarget{put-any-text-you-may-want-to-type-here.-23}{%
\paragraph{Put any text you may want to type
here.}\label{put-any-text-you-may-want-to-type-here.-23}}

\hypertarget{end-of-text-block.-25}{%
\paragraph{End of text block.}\label{end-of-text-block.-25}}

\begin{enumerate}
\def\labelenumi{\Alph{enumi})}
\setcounter{enumi}{3}
\tightlist
\item
  Execute the code block below. Don't edit it! This creates a heatmap
  showing only the 20 metabolites whose mean abundance in your chosen
  conditions differs the most from the control sample (21-L timepoint
  0). Compare the result with figure 1 from the paper. Can you see a
  clear difference in metabolite abundance between your two chosen
  conditions? Check the biological role of the selected metabolites.
  Does the selection of metabolites make biological sense given your
  choice of two conditions to compare? \textbf{{[}6 points{]}}
\end{enumerate}

\begin{Shaded}
\begin{Highlighting}[]
\FunctionTok{make\_heatmap}\NormalTok{(df\_metabolomics, condition1, condition2)}
\end{Highlighting}
\end{Shaded}

\begin{verbatim}
## Warning in heatmap.2a(as.matrix(data[, order(conditionLabels)]), scale =
## "row", : Using scale="row" or scale="column" when breaks arespecified can
## produce unpredictable results.Please consider using only one or the other.
\end{verbatim}

\includegraphics{BDA-project-2022-2023_files/figure-latex/Assignment 4D-1.pdf}

\hypertarget{put-any-text-you-may-want-to-type-here.-24}{%
\paragraph{Put any text you may want to type
here.}\label{put-any-text-you-may-want-to-type-here.-24}}

\hypertarget{end-of-text-block.-26}{%
\paragraph{End of text block.}\label{end-of-text-block.-26}}

\begin{enumerate}
\def\labelenumi{\Alph{enumi})}
\setcounter{enumi}{4}
\tightlist
\item
  Create a UMAP of both datasets. Supply the custom.config as argument
  to the umap() function when running it. Select an appropriate number
  of neighbors. Plot the result. Add a legend to identify the colour for
  each condition. Does UMAP find the same kind of clusters as the
  hierarchical clustering or the PCA? What can you say about the
  metabolites that are important for the UMAP clustering? \textbf{{[}8
  points{]}}
\end{enumerate}

\begin{Shaded}
\begin{Highlighting}[]
\NormalTok{custom.config }\OtherTok{=}\NormalTok{ umap.defaults}
\NormalTok{custom.config}\SpecialCharTok{$}\NormalTok{random\_state }\OtherTok{=} \DecValTok{123}
\end{Highlighting}
\end{Shaded}

\hypertarget{put-any-text-you-may-want-to-type-here.-25}{%
\paragraph{Put any text you may want to type
here.}\label{put-any-text-you-may-want-to-type-here.-25}}

\hypertarget{end-of-text-block.-27}{%
\paragraph{\texorpdfstring{End of text block.\\
}{End of text block. }}\label{end-of-text-block.-27}}

\hfill\break
\textbf{This concludes Assignment 4. Make sure you save your .Rmd file
and workspace so you can continue in your next project session.}

\hypertarget{assignment-5-classification-johan-westerhuis-19-01-2023}{%
\subsubsection{Assignment 5: Classification (Johan Westerhuis,
19-01-2023)}\label{assignment-5-classification-johan-westerhuis-19-01-2023}}

In this assignment we will use the LDA and PCDA methods to find the
metabolites that best discriminate between your two chosen conditions.\\

\begin{enumerate}
\def\labelenumi{\Alph{enumi})}
\tightlist
\item
  Make a reduced metabolomics dataset by selecting only the samples of
  your two chosen conditions. If one of your conditions is ``21-L'',
  remove the 21-L sample of timepoint 0 to balance your groups. Create a
  vector of binary values encoding the condition of each sample (0 =
  condition1 and 1 = condition2). (Hint: consider recentering and/or
  rescaling the reduced dataset.) \textbf{{[}5 points{]}}
\end{enumerate}

\hypertarget{put-any-text-you-may-want-to-type-here.-26}{%
\paragraph{Put any text you may want to type
here.}\label{put-any-text-you-may-want-to-type-here.-26}}

\hypertarget{end-of-text-block.-28}{%
\paragraph{End of text block.}\label{end-of-text-block.-28}}

\begin{enumerate}
\def\labelenumi{\Alph{enumi})}
\setcounter{enumi}{1}
\tightlist
\item
  Perform a linear discriminant analysis on your new metabolomics
  dataset, using your new vector of binary values as the class of each
  sample. \textbf{{[}2 points{]}}
\end{enumerate}

\hypertarget{put-any-text-you-may-want-to-type-here.-27}{%
\paragraph{Put any text you may want to type
here.}\label{put-any-text-you-may-want-to-type-here.-27}}

\hypertarget{end-of-text-block.-29}{%
\paragraph{End of text block.}\label{end-of-text-block.-29}}

\begin{enumerate}
\def\labelenumi{\Alph{enumi})}
\setcounter{enumi}{2}
\tightlist
\item
  You should have gotten a warning from your LDA function stating that
  the variables in the reduced dataset are collinear. What does this
  warning mean? Do you think you have created a valid model? If so,
  analyse it for important metabolites. If not, explain why the model is
  not valid. \textbf{{[}6 points{]}}
\end{enumerate}

\hypertarget{put-any-text-you-may-want-to-type-here.-28}{%
\paragraph{Put any text you may want to type
here.}\label{put-any-text-you-may-want-to-type-here.-28}}

\hypertarget{end-of-text-block.-30}{%
\paragraph{End of text block.}\label{end-of-text-block.-30}}

\begin{enumerate}
\def\labelenumi{\Alph{enumi})}
\setcounter{enumi}{3}
\tightlist
\item
  Make a principal component analysis of your reduced metabolomics
  dataset. Only consider 2 components. Show the amount of variance
  explained per component in the plot labels. Add a legend to identify
  the colour for each condition. Does your PCA model separate your two
  chosen conditions well? (Hint: use svd(). Consider recentering and/or
  rescaling your data for PCA specifically!) \textbf{{[}7 points{]}}
\end{enumerate}

\hypertarget{put-any-text-you-may-want-to-type-here.-29}{%
\paragraph{Put any text you may want to type
here.}\label{put-any-text-you-may-want-to-type-here.-29}}

\hypertarget{end-of-text-block.-31}{%
\paragraph{End of text block.}\label{end-of-text-block.-31}}

\begin{enumerate}
\def\labelenumi{\Alph{enumi})}
\setcounter{enumi}{4}
\tightlist
\item
  Make a PCDA model using the PCA scores (from question D) for the
  discriminant analysis. Inspect the classifications of your samples.
  Does the model misclassify any of the samples? What is the most
  important metabolite that differentiates your two chosen conditions?
  \textbf{{[}9 points{]}}
\end{enumerate}

\hypertarget{put-any-text-you-may-want-to-type-here.-30}{%
\paragraph{Put any text you may want to type
here.}\label{put-any-text-you-may-want-to-type-here.-30}}

\hypertarget{end-of-text-block.-32}{%
\paragraph{End of text block.}\label{end-of-text-block.-32}}

\begin{enumerate}
\def\labelenumi{\Alph{enumi})}
\setcounter{enumi}{5}
\tightlist
\item
  Compare your metabolomics PCDA result with your metabolomics PCA
  result (from assignment 3C). Do you find the same important
  metabolites? If you find any differences, why would the selected
  metabolites be different? \textbf{{[}5 points{]}}
\end{enumerate}

\hypertarget{put-any-text-you-may-want-to-type-here.-31}{%
\paragraph{Put any text you may want to type
here.}\label{put-any-text-you-may-want-to-type-here.-31}}

\hypertarget{end-of-text-block.-33}{%
\paragraph{\texorpdfstring{End of text block.\\
}{End of text block. }}\label{end-of-text-block.-33}}

\hfill\break
\textbf{This concludes Assignment 5. Make sure you save your .Rmd file
and workspace so you can continue in your next project session.}

\hypertarget{assignment-6-asca-age-smilde-20-01-2023}{%
\subsubsection{Assignment 6: ASCA (Age Smilde,
20-01-2023)}\label{assignment-6-asca-age-smilde-20-01-2023}}

In this assignment we want you to use the shiny app for ASCA with the
Caldana et al.~metabolomics data. To answer the questions below, just
make a screenshot of your plot window and include it into the markdown
as follows (see the .Rmd):

Open up the shiny app for ASCA using the DataTool.R script in the /Shiny
App/ folder. From the /Data/ folder, load as your dataset the file
``Data\_for\_ASCA.csv'' and as your design file
``Design\_for\_ASCA.csv''. The Data\_for\_ASCA.csv file contains
untransformed metabolomics data. We have selected only the 21-D, 21-L,
32-D, 32-L, 4-D and 4-L conditions for you to use in this assignment.
There are 4 replicates present for every condition-timepoint
combination. The Design\_for\_ASCA.csv file contains the time,
temperature and light condition metadata for every sample. These are
encoded as factors 1, 2 and 3, respectively.

\begin{enumerate}
\def\labelenumi{\Alph{enumi})}
\tightlist
\item
  Consider balancing, transforming, centering and/or scaling the data in
  the shiny app. Defend your choices. Inspect the PCA plots and change
  the point colours to light, time or temperature. Does the PCA model
  separate the different groups well? Which metabolites are important
  for this separation? Include screenshots of your score plot and
  biplot. \textbf{{[}7 points{]}}
\end{enumerate}

\hypertarget{include-your-screenshots-here.}{%
\paragraph{\texorpdfstring{Include your screenshots here.\\
}{Include your screenshots here. }}\label{include-your-screenshots-here.}}

\hypertarget{end-of-screenshot-block.}{%
\paragraph{\texorpdfstring{End of screenshot block.\\
}{End of screenshot block. }}\label{end-of-screenshot-block.}}

\hfill\break

\hypertarget{put-any-text-you-may-want-to-type-here.-32}{%
\paragraph{Put any text you may want to type
here.}\label{put-any-text-you-may-want-to-type-here.-32}}

\hypertarget{end-of-text-block.-34}{%
\paragraph{End of text block.}\label{end-of-text-block.-34}}

\begin{enumerate}
\def\labelenumi{\Alph{enumi})}
\setcounter{enumi}{1}
\tightlist
\item
  Go to the univariate plot window. Plot the metabolites that were
  important according to your metabolomics PCDA result from Assignment
  5. Why do you think these metabolites were selected? Include a
  screenshot of your univariate plots. \textbf{{[}4 points{]}}
\end{enumerate}

\hypertarget{include-your-screenshots-here.-1}{%
\paragraph{\texorpdfstring{Include your screenshots here.\\
}{Include your screenshots here. }}\label{include-your-screenshots-here.-1}}

\hypertarget{end-of-screenshot-block.-1}{%
\paragraph{\texorpdfstring{End of screenshot block.\\
}{End of screenshot block. }}\label{end-of-screenshot-block.-1}}

\hfill\break

\hypertarget{put-any-text-you-may-want-to-type-here.-33}{%
\paragraph{Put any text you may want to type
here.}\label{put-any-text-you-may-want-to-type-here.-33}}

\hypertarget{end-of-text-block.-35}{%
\paragraph{End of text block.}\label{end-of-text-block.-35}}

\begin{enumerate}
\def\labelenumi{\Alph{enumi})}
\setcounter{enumi}{2}
\tightlist
\item
  In the ASCA window of the shiny app, create a model using the factors
  time, light, and temperature. Do not consider interaction terms yet.
  Examine the model. Which metabolites are important in separating the
  different temperature conditions? Which metabolites are important in
  separating the different light conditions? Have you found these
  metabolites before using other methods? Include screenshots of your
  score plots and biplots. \textbf{{[}3 points{]}}
\end{enumerate}

\hypertarget{include-your-screenshots-here.-2}{%
\paragraph{\texorpdfstring{Include your screenshots here.\\
}{Include your screenshots here. }}\label{include-your-screenshots-here.-2}}

\hypertarget{end-of-screenshot-block.-2}{%
\paragraph{\texorpdfstring{End of screenshot block.\\
}{End of screenshot block. }}\label{end-of-screenshot-block.-2}}

\hfill\break

\hypertarget{put-any-text-you-may-want-to-type-here.-34}{%
\paragraph{Put any text you may want to type
here.}\label{put-any-text-you-may-want-to-type-here.-34}}

\hypertarget{end-of-text-block.-36}{%
\paragraph{End of text block.}\label{end-of-text-block.-36}}

\begin{enumerate}
\def\labelenumi{\Alph{enumi})}
\setcounter{enumi}{3}
\tightlist
\item
  Use the ``Combine terms'' text field to create a new ASCA model
  considering a light factor term and an interaction term of light and
  temperature. Examine the model in the Combination plots window. Have a
  look at the levels plot. What is the overall behaviour of the
  metabolites between the different temperature conditions? Is the
  overall behaviour different for the light and dark samples? Include a
  screenshot of your levels plot. \textbf{{[}5 points{]}}
\end{enumerate}

\hypertarget{include-your-screenshots-here.-3}{%
\paragraph{\texorpdfstring{Include your screenshots here.\\
}{Include your screenshots here. }}\label{include-your-screenshots-here.-3}}

\hypertarget{end-of-screenshot-block.-3}{%
\paragraph{\texorpdfstring{End of screenshot block.\\
}{End of screenshot block. }}\label{end-of-screenshot-block.-3}}

\hfill\break

\hypertarget{put-any-text-you-may-want-to-type-here.-35}{%
\paragraph{Put any text you may want to type
here.}\label{put-any-text-you-may-want-to-type-here.-35}}

\hypertarget{end-of-text-block.-37}{%
\paragraph{End of text block.}\label{end-of-text-block.-37}}

\begin{enumerate}
\def\labelenumi{\Alph{enumi})}
\setcounter{enumi}{4}
\tightlist
\item
  In the ASCA window of the shiny app, use the ``Combine terms'' text
  field to create a new ASCA model considering a time factor term and an
  interaction term of time and light. Inspect the levels plot of the
  model. What is the overall behaviour of the metabolites over time? Is
  that behaviour different for the different temperature conditions?
  Include a screenshot of your levels plot. \textbf{{[}5 points{]}}
\end{enumerate}

\hypertarget{include-your-screenshots-here.-4}{%
\paragraph{\texorpdfstring{Include your screenshots here.\\
}{Include your screenshots here. }}\label{include-your-screenshots-here.-4}}

\hypertarget{end-of-screenshot-block.-4}{%
\paragraph{\texorpdfstring{End of screenshot block.\\
}{End of screenshot block. }}\label{end-of-screenshot-block.-4}}

\hfill\break

\hypertarget{put-any-text-you-may-want-to-type-here.-36}{%
\paragraph{Put any text you may want to type
here.}\label{put-any-text-you-may-want-to-type-here.-36}}

\hypertarget{end-of-text-block.-38}{%
\paragraph{End of text block.}\label{end-of-text-block.-38}}

\begin{enumerate}
\def\labelenumi{\Alph{enumi})}
\setcounter{enumi}{5}
\tightlist
\item
  Compare all of your ``most important'' metabolites from all of the
  methods you have applied. What are the similarities? What are the
  differences? What is causing some methods to produce different
  results? Which results are the most valid, in your opinion? Which
  results make the most biological sense? \textbf{{[}9 points{]}}
\end{enumerate}

\hypertarget{put-any-text-you-may-want-to-type-here.-37}{%
\paragraph{Put any text you may want to type
here.}\label{put-any-text-you-may-want-to-type-here.-37}}

\hypertarget{end-of-text-block.-39}{%
\paragraph{End of text block.}\label{end-of-text-block.-39}}

\hfill\break
\hfill\break
\textbf{That concludes week 2 of the project! Please hand in your .rmd
and .pdf files on Canvas for your project grade. The week 2 deadline is
Monday, January 20th 11:00.}

\end{document}
